\documentclass[12pt,letterpaper,oneside,reqno]{amsart}
\usepackage{amsfonts}
\usepackage{amsmath}
\usepackage{amssymb}
\usepackage{amsthm}
\usepackage{float}
\usepackage{mathrsfs}
\usepackage{colonequals}
\usepackage[font=small,labelfont=bf]{caption}
\usepackage[left=1in,right=1in,bottom=1in,top=1in]{geometry}
\usepackage[pdfpagelabels,hyperindex,colorlinks=true,linkcolor=blue,urlcolor=magenta,citecolor=green]{hyperref}
\usepackage{setspace}
\onehalfspacing
\emergencystretch=1em

\newcommand \anglePower [2]{\langle #1 \rangle \sp{#2}}
\newcommand \bernoulli [2][B] {{#1}\sb{#2}}
\newcommand \curvePower [2]{\{#1\}\sp{#2}}
\newcommand \coeffA [3][A] {{\mathbf{#1}} \sb{#2,#3}}
\newcommand \polynomialP [4][P]{{\mathbf{#1}}\sp{#2} \sb{#3}(#4)}

% ordinary derivatives
\newcommand \derivative [2] {\frac{d}{d #2} #1}                              % 1 - function; 2 - variable;
\newcommand \pderivative [2] {\frac{\partial #1}{\partial #2}}               % 1 - function; 2 - variable;
\newcommand \qderivative [1] {D_{q} #1}                                      % 1 - function
\newcommand \nqderivative [1] {D_{n,q} #1}                                   % 1 - function
\newcommand \qpowerDerivative [1] {\mathcal{D}_q #1}                         % 1 - function;
\newcommand \finiteDifference [1] {\Delta #1}                                % 1 - function;
\newcommand \pTsDerivative [2] {\frac{\partial #1}{\Delta #2}}               % 1 - function; 2 - variable;

% high order derivatives
\newcommand \derivativeHO [3] {\frac{d^{#3}}{d {#2}^{#3}} #1}                % 1 - function; 2 - variable; 3 - order
\newcommand \pderivativeHO [3]{\frac{\partial^{#3}}{\partial {#2}^{#3}} #1}
\newcommand \qderivativeHO [2] {D_{q}^{#2} #1}                               % 1 - function; 2 - order
\newcommand \qpowerDerivativeHO [2] {\mathcal{D}_{q}^{#2} #1}                % 1 - function; 2 - order
\newcommand \finiteDifferenceHO [2] {\Delta^{#2} #1}                         % 1 - function; 2 - order
\newcommand \pTsDerivativeHO [3] {\frac{\partial^{#3}}{\Delta {#2}^{#3}} #1} % 1 - function; 2 - variable;

\newtheorem{thm}{Theorem}[section]
\newtheorem{cor}[thm]{Corollary}
\newtheorem{lem}[thm]{Lemma}
\newtheorem{examp}[thm]{Example}
\newtheorem{definition}[thm]{Definition}

\numberwithin{equation}{section}

\title[Polynomial identities involving Pascal's triangle rows]
{Polynomial identities involving Pascal's triangle rows}
\author[Petro Kolosov]{Petro Kolosov}
\email{kolosovp94@gmail.com}
\keywords{
    Binomial coefficients, Binomial theorem, Pascal's triangle, Binomial sums, Binomial distribution, Binomial identities}
\urladdr{https://razumovsky.me/}
\subjclass[2010]{26E70, 05A30}
\date{\today}
\hypersetup{
    pdftitle={Polynomial identities involving Pascal's triangle rows},
    pdfsubject={
        Binomial coefficients, Binomial theorem, Pascal's triangle, Binomial sums, Binomial distribution, Binomial identities,
        Polynomial identities
    },
    pdfauthor={Petro Kolosov},
    pdfkeywords={
        Binomial coefficients, Binomial theorem, Pascal's triangle, Binomial sums, Binomial distribution, Binomial identities,
        Polynomial identities
    }
}
\begin{document}
    \begin{abstract}
        In this short report we consider the famous binomial identity
\[
    \sum_{k=0}^{n}\binom{n}{k}=2^n
\]
Based on it, the following binomial identities are derived
\begin{gather*}
    m^n = \sum_{k=0}^{n} \sum_{j=0}^{k} \binom{n}{k} \binom{k}{j} (-1)^{k-j} m^j,\\
    m^n = \sum_{k=0}^{n} \sum_{j=0}^{k} \binom{n}{j} \binom{n-j}{k-j} (-1)^{k-j} m^j,\\
\end{gather*}
where $\binom{n}{k}$ are binomial coefficients and $(m, \ n)$ are non-negative integers.

    \end{abstract}

    \maketitle

    \tableofcontents


    \section{Introduction} \label{sec:introduction}
    We start from the famous relation about row sums of the Pascal triangle, that is
\begin{equation}
    \sum_{k=0}^{n}\binom{n}{k} = 2^n,
    \label{eq:pascal_triangle_row_sums}
\end{equation}
where $\binom{n}{k} = \frac{n!}{k!(n-k)!}$ are binomial coefficients~\cite{graham1989concrete}.
Identity~\eqref{eq:pascal_triangle_row_sums} is straightforward because the Pascal's triangle is
\begin{table}[H]
    \begin{tabular}{c|ccccccccc}
        $n/k$ & 0 & 1 & 2  & 3  & 4  & 5  & 6  & 7 & 8 \\ [3px]
        \hline
        0     & 1 &   &    &    &    &    &    &   &   \\
        1     & 1 & 1 &    &    &    &    &    &   &   \\
        2     & 1 & 2 & 1  &    &    &    &    &   &   \\
        3     & 1 & 3 & 3  & 1  &    &    &    &   &   \\
        4     & 1 & 4 & 6  & 4  & 1  &    &    &   &   \\
        5     & 1 & 5 & 10 & 10 & 5  & 1  &    &   &   \\
        6     & 1 & 6 & 15 & 20 & 15 & 6  & 1  &   &   \\
        7     & 1 & 7 & 21 & 35 & 35 & 21 & 7  & 1 &   \\
        8     & 1 & 8 & 28 & 56 & 70 & 56 & 21 & 8 & 1 \\
    \end{tabular}
    \caption{Pascal's triangle~\cite{conway1996pascal}.
    Each $k-$th term of $n-$th row is $\binom{n}{k}\cdot 1^k$.
    Sequence \href{https://oeis.org/A007318}{\texttt{A007318}} in OEIS~\cite{Sloane_theencyclopedia}.} \label{tab:pascal_triagnle}
\end{table}
Consider a generating function such as $f_2(n,k) = \binom{n}{k}\cdot 2^k$.
The function $f_2(n,k)$ generates the following Pascal-like triangle
\begin{table}[H]
    \begin{tabular}{c|ccccccccc}
        $n/k$ & 0 & 1  & 2   & 3   & 4    & 5    & 6    & 7    & 8 \\ [3px]
        \hline
        0     & 1 &    &     &     &      &      &      &      &     \\
        1     & 1 & 2  &     &     &      &      &      &      &     \\
        2     & 1 & 4  & 4   &     &      &      &      &      &     \\
        3     & 1 & 6  & 12  & 8   &      &      &      &      &     \\
        4     & 1 & 8  & 24  & 32  & 16   &      &      &      &     \\
        5     & 1 & 10 & 40  & 80  & 80   & 32   &      &      &     \\
        6     & 1 & 12 & 60  & 160 & 240  & 192  & 64   &      &     \\
        7     & 1 & 14 & 84  & 280 & 560  & 672  & 448  & 128  &     \\
        8     & 1 & 16 & 112 & 448 & 1120 & 1792 & 1792 & 1024 & 256 \\
    \end{tabular}
    \caption{Triangle generated by the function $\binom{n}{k}\cdot 2^k$.
    Can be reproduced using Mathematica function \texttt{GeneratePascalLikeTriangle[2, 8]} at~\cite{PK22Source}.
    Sequence \href{https://oeis.org/A013609}{\texttt{A013609}} in OEIS~\cite{Sloane_theencyclopedia}.}
    \label{tab:third_power}
\end{table}
Now we can notice that
\begin{equation}
    \sum_{k=0}^{n} \binom{n}{k} \cdot 2^k = 3^n\label{eq:third_power}
\end{equation}
Continue similarly we can generalize the equations~\eqref{eq:pascal_triangle_row_sums},~\eqref{eq:third_power} as follows
\begin{align*}
    2^n &= \sum_{k=0}^{n}\binom{n}{k} \cdot 1^k \\
    3^n &= \sum_{k=0}^{n}\binom{n}{k} \cdot 2^k \\
    4^n &= \sum_{k=0}^{n}\binom{n}{k} \cdot 3^k \\
    &\dots \\
    m^n &= \sum_{k=0}^{n}\binom{n}{k} \cdot (m-1)^k
\end{align*}
Therefore, let be a theorem
\begin{thm}
    The following identity involving polynomial $m^n$ holds
    \begin{equation}
        m^n=\sum_{k=0}^{n}\sum_{j=0}^{k}\binom{n}{k}\binom{k}{j}(-1)^{k-j}m^j\label{eq:theorem-1}
    \end{equation}
    where $(m, n)$ are non-negative integers.
    \begin{proof}
        Recall the induction over $m$, let be a base case $m=2$, hereby
        \begin{equation}
            2^n=\sum_{k=0}^{n}\binom{n}{k}(2-1)^k\label{eq:induction-2}
        \end{equation}
        Reviewing an equation~\eqref{eq:induction-2} we can see that
        \begin{equation}
        (\underbrace{2+1}_{m=3})
            ^n=\sum_{k=0}^{n}\binom{n}{k}\cdot (\underbrace{(2-1)+1}_{m-1})^k\label{eq:induction-3}
        \end{equation}
        Continue similarly it is straightforward that $m^n = \sum_{k=0}^{n}\binom{n}{k} \cdot (m-1)^k$.
        However, we are able to expand the part $(m-1)^k$ by means of Binomial theorem~\cite{AbraSteg72}, that is
        \[
            (m-1)^k = \sum_{j=0}^{k} \binom{k}{j} (-1)^{k-j} m^j
        \]
        So that now we are able to merge both results $m^n = \sum_{k=0}^{n}\binom{n}{k} \cdot (m-1)^k$
        and $(m-1)^k = \sum_{j=0}^{k} \binom{k}{j} (-1)^{k-j} m^j$ to receive
        \begin{align*}
            \label{hyp8}
            m^n
            &=& \sum_{k=0}^{n} \binom{n}{k} \cdot \underbrace{(m-1)^k}_{\sum_{j=0}^{k} \binom{k}{j} (-1)^{k-j} m^j}
            =\sum_{k=0}^{n} \binom{n}{k} \sum_{j=0}^{k} \binom{k}{j} (-1)^{k-j} m^j\\
            &=& \sum_{k=0}^{n} \sum_{j=0}^{k} \binom{n}{k} \binom{k}{j} (-1)^{k-j} m^j
        \end{align*}
        Theorem~\eqref{eq:theorem-1} may be verified using Mathematica command \texttt{PolynomialIdentity[m, n]}
        at~\cite{PK22Source}.
        This completes the proof.
    \end{proof}
\end{thm}
Moreover, by means of the binomial identity [\cite{gross2016combinatorial}, Chapter 4]
\[
    \binom{n}{k} \binom{k}{j} = \binom{n}{j} \binom{n-j}{k-j}
\]
The polynomial $m^n$ is identical to
\[
    m^n = \sum_{k=0}^{n} \sum_{j=0}^{k} \binom{n}{j} \binom{n-j}{k-j} (-1)^{k-j} m^j
\]


    \section{Conclusions}\label{sec:conclusions}
    The following binomial identities are derived
\begin{gather*}
    m^n = \sum_{k=0}^{n} \sum_{j=0}^{k} \binom{n}{k} \binom{k}{j} (-1)^{k-j} m^j,\\
    m^n = \sum_{k=0}^{n} \sum_{j=0}^{k} \binom{n}{j} \binom{n-j}{k-j} (-1)^{k-j} m^j.\\
\end{gather*}
Moreover, above results are verified by means of specified Mathematica scripts available at
\href{https://github.com/kolosovpetro/PolynomialIdentitiesInvolvingPascalsTriangleRows/blob/develop/mathematica/PolynomialIdentitiesInvolvingPascalsTriangleRows.m}
{\texttt{github.com/kolosovpetro/PolynomialIdentitiesInvolvingPascalsTriangleRows}}.


    \bibliographystyle{alpha}
    \bibliography{PolynomialIdentitiesInvolvingPascalsTriangleRows}

\end{document}
