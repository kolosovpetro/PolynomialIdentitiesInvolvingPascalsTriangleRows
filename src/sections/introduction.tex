We start from the famous relation about row sums of the Pascal triangle, that is
\begin{equation}
    \sum_{k=0}^{n}\binom{n}{k} = 2^n,
    \label{eq:pascal_triangle_row_sums}
\end{equation}
where $\binom{n}{k} = \frac{n!}{k!(n-k)!}$ are binomial coefficients~\cite{graham1989concrete}.
Identity~\eqref{eq:pascal_triangle_row_sums} is straightforward because the Pascal's triangle is
\begin{table}[H]
    \begin{tabular}{c|ccccccccc}
        $n/k$ & 0 & 1 & 2  & 3  & 4  & 5  & 6  & 7 & 8 \\ [3px]
        \hline
        0     & 1 &   &    &    &    &    &    &   &   \\
        1     & 1 & 1 &    &    &    &    &    &   &   \\
        2     & 1 & 2 & 1  &    &    &    &    &   &   \\
        3     & 1 & 3 & 3  & 1  &    &    &    &   &   \\
        4     & 1 & 4 & 6  & 4  & 1  &    &    &   &   \\
        5     & 1 & 5 & 10 & 10 & 5  & 1  &    &   &   \\
        6     & 1 & 6 & 15 & 20 & 15 & 6  & 1  &   &   \\
        7     & 1 & 7 & 21 & 35 & 35 & 21 & 7  & 1 &   \\
        8     & 1 & 8 & 28 & 56 & 70 & 56 & 21 & 8 & 1 \\
    \end{tabular}
    \caption{Pascal's triangle~\cite{conway1996pascal}.
    Each $k-$th term of $n-$th row is $\binom{n}{k}\cdot 1^k$.} \label{tab:pascal_triagnle}
\end{table}
Consider a generating function such as $f_2(n,k) = \binom{n}{k}\cdot 2^k$.
The function $f_2(n,k)$ generates the following Pascal-like triangle
\begin{table}[H]
    \begin{tabular}{c|ccccccccc}
        $n/k$ & 0 & 1  & 2   & 3   & 4    & 5    & 6    & 7    & 8 \\ [3px]
        \hline
        0     & 1 &    &     &     &      &      &      &      &     \\
        1     & 1 & 2  &     &     &      &      &      &      &     \\
        2     & 1 & 4  & 4   &     &      &      &      &      &     \\
        3     & 1 & 6  & 12  & 8   &      &      &      &      &     \\
        4     & 1 & 8  & 24  & 32  & 16   &      &      &      &     \\
        5     & 1 & 10 & 40  & 80  & 80   & 32   &      &      &     \\
        6     & 1 & 12 & 60  & 160 & 240  & 192  & 64   &      &     \\
        7     & 1 & 14 & 84  & 280 & 560  & 672  & 448  & 128  &     \\
        8     & 1 & 16 & 112 & 448 & 1120 & 1792 & 1792 & 1024 & 256 \\
    \end{tabular}
    \caption{Triangle generated by the function $\binom{n}{k}\cdot 2^k$.
    Can be reproduced using Mathematica function \texttt{GeneratePascalLikeTriangle[2, 8]} at~\cite{PK22Source}.}
    \label{tab:third_power}
\end{table}
Now we can notice that
\begin{equation}
    \sum_{k=0}^{n} \binom{n}{k} \cdot 2^k = 3^n\label{eq:third_power}
\end{equation}
Continue similarly we can generalize the equations~\eqref{eq:pascal_triangle_row_sums},~\eqref{eq:third_power} as follows
\begin{align*}
    2^n &= \sum_{k=0}^{n}\binom{n}{k} \cdot 1^k \\
    3^n &= \sum_{k=0}^{n}\binom{n}{k} \cdot 2^k \\
    4^n &= \sum_{k=0}^{n}\binom{n}{k} \cdot 3^k \\
    &\dots \\
    m^n &= \sum_{k=0}^{n}\binom{n}{k} \cdot (m-1)^k
\end{align*}
\begin{thm}
    Volume of \textit{n}-dimension hypercube with length \textit{m} could be calculated as
    \begin{equation}
        \label{hyp3}
        m^n=\sum_{k=0}^{n}\sum_{j=0}^{k}\binom{n}{k}\binom{k}{j}(-1)^{k-j}m^j
    \end{equation}
    where m and n - positive integers, see \cite{10}.
    \begin{proof}
        Recall induction over \textit{m}, in (\hyperref[hyp1]{1.1}) is shown a well-known example for $m=2$.
        \begin{equation}
            \label{d1}
            2^n=\sum_{k=0}^{n}\binom{n}{k}(2-1)^k
        \end{equation}
        Review (\hyperref[d1]{1.5}) and suppose that
        \begin{equation}
            \label{hyp4}
            (\underbrace{2+1}_{m=3})^n=\sum_{k=0}^{n}\binom{n}{k}\cdot (\underbrace{(2-1)+1}_{m-1})^k
        \end{equation}
        And, obviously, this statement holds by means of Newton's Binomial Theorem \cite{2}, \cite{4} given $m=3$, more detailed, recall expansion for $(x+1)^n$ to show it.
        \begin{equation}
            \label{d2}
            (x+1)^n=\sum_{k=0}^{n}\binom{n}{k}x^k
        \end{equation}
        Substituting $x=2$ to (\hyperref[d2]{1.7}) we have reached (\hyperref[hyp4]{1.6}).\\
        Next, let show example for each $m\in\mathbb{N}$. Recall Binomial theorem to show this
        \begin{equation}
            \label{hyp5}
            m^n=\sum_{k=0}^{n}\binom{n}{k}\cdot (m-1)^k
        \end{equation}
        Hereby, for $m+1$ we receive Binomial theorem again
        \begin{equation}
            \label{hyp6}
            (m+1)^n=\sum_{k=0}^{n}\binom{n}{k}\cdot m^k
        \end{equation}
        Review result from (\hyperref[hyp5]{1.8}) and substituting Binomial expansion $\sum_{j=0}^{k}\binom{k}{j}(-1)^{n-k}m^j$ instead $(m-1)^k$ we receive desired result
        \begin{eqnarray}
            \label{hyp8}
            m^n &=& \sum_{k=0}^{n}\binom{n}{k}\cdot \underbrace{(m-1)^k}_{\sum_{j=0}^{k}\binom{k}{j}(-1)^{k-j}m^j}=\sum_{k=0}^{n}\binom{n}{k}\sum_{j=0}^{k}\binom{k}{j}(-1)^{k-j}m^j\\
            &=& \sum_{k=0}^{n}\sum_{j=0}^{k}\binom{n}{k}\binom{k}{j}(-1)^{k-j}m^j \nonumber
        \end{eqnarray}
        This completes the proof.
    \end{proof}
\end{thm}
\begin{lem}
    Number of elements $k-$face elements $\mathscr{E}_k(\mathrm{\mathbf{Y}}_n^p)$ of Generalized Hypercube $\mathrm{\mathbf{Y}}_n^p$ equals to
    \begin{equation}
        \label{l1}
        \mathscr{E}_k(\mathrm{\mathbf{Y}}_n^p)=\sum_{j=0}^{k}\binom{n}{k}\binom{k}{j}(-1)^{k-j}(p-1)^j
    \end{equation}
\end{lem}
Expression (\hyperref[hyp8]{1.10}) has Multinomial analog
\begin{equation}
    \label{m1}
    \sum_{a_1+a_2+\cdots+a_k=n}\binom{n}{a_1,a_2,\ldots,a_k}=k^n
\end{equation}
%To check this expression (\hyperref[hyp8]{1.10}) use Mathematica code
%\begin{spverbatim}
%    In[19]:= Sum[Sum[Binomial[n,k]*Binomial[k,j]*(-1)^(k-j)*m^j,
%        {j, 0, k}],{k, 0, n}]
%\end{spverbatim}
It could be found at .txt file in the last line \href{https://kolosovpetro.github.io/mathematica_codes/Series_Representation_Mathematica_codes.txt}{here}.